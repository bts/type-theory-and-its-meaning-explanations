\documentclass[main.tex]{subfiles}

\begin{document}

\onehalfspacing

\chapter{Proof Theories and Logical Frameworks}

\section{Proof-Theoretic Type Theory}

\subsection{Analytic and Synthetic Judgement}

A synthetic judgement is one for which the experience of \emph{coming to know
it} necessarily results in the synthesis of some new information; on the other
hand, to know an \emph{analytic} judgement is to know it purely on the basis of
the information contained inside it. So analytic judgements are decidable,
since if they may become evident, it will be purely on the basis of their own
content; whereas synthetic judgements become evident when one has acquired some
particular evidence for them.

A logical theory has, then, both analytic and synthetic judgements; the
judgement \isprop{P} is analytic, since its evidence follows from the
definition of $P$, whereas the assertion of \istrue{P} entails the knowledge of
some extra information, namely a verification of $P$. When we have extended the
logical theory to a type theory in the manner of the previous sections, the
judgement \ver{M}{P} is also synthetic, since \ver{M}{P} is not self-evident in
general.

But why is it not enough to assert that $M$ verifies $P$ to know \ver{M}{P}? It
suffices to define a $P$ such that one cannot decide in general whether some
term is a verification of it. Let us define the propositional symbol
$\mathsf{P}$, and we intend to know the judgement \isprop{\mathsf{P}}, whose
meaning is to be expanded as follows:
\begin{quote}
  To know \isprop{\mathsf{P}} is to know counts as a canonical verification of $\mathsf{P}$.
\end{quote}

We will say, then, that $\bullet$ is a canonical verification of $\mathsf{P}$
just when Goldbach's conjecture is true. Then it comes immediately that the
judgement \ver{M}{\mathsf{P}} may not be known or refuted on its own basis, nor
even the judgement \ver{\bullet}{\mathsf{P}}, since they depend on an
proposition whose truth is not known:

\begin{itemize}
  \item[] To know \ver{M}{\mathsf{P}} is to know that \reduce{M}{M'} to a
    canonical verification of $P$.
  \item[$\Leftrightarrow$] To know $\ver{M}{\mathsf{P}}$ is to know that
    \reduce{M}{\bullet} such that Goldbach's conjecture is true.
\end{itemize}

\subsection{Proof of a judgement vs.\ verification of a proposition}

Because the judgement \ver{M}{P} is synthetic, we cannot say that it gives rise
to a proof theory for the logic, since the core judgement of a proof theory
$M:A$ must be analytic, in order to avoid the infinite regress of a proof
theory requiring a proof theory requiring a proof theory, and so on.

The notion of verification of a proposition could never be the same as proof
anyway, except in the most trivial circumstances, since a verification is meant
to be an effective operation which realizes the truth of the proposition, and
no constraints whatsoever (termination, totality, etc.) are placed on these
operations except those which come from the meaning of the judgements (see
Dummett).

So a proof theory is necessarily intensional, and its judgements are to be
analytic/decidable. What is it, then, that we have considered so far which
corresponds with a proof $M$ such that $M:P$ in a proof theory? As discussed
above, $M$ is not merely a term such that \ver{M}{P}, since this is not in
general enough information to know whether $M$ is a proof. In fact, $M$ must
comprise all the logical inferences which led to the knowledge that $P$ is
true, and so a meaning explanation for the judgement $M:P$ in a proof theory
immediately suggests itself:
\begin{quote}
  To know $M:P$ is to know that $M$ is evidence of the judgement \istrue{P}.
\end{quote}

And so the term domain of the proof theory is not the same as the one that we
have considered so far; it must consist in terms which represent traces of the
inferences made in the course of knowing the judgements of a logical theory.
There is a sense in which one can consider the types of a proof theory to
interpret the judgements of the logical theory, and this methodology is called
``judgements as types'' (and this implies ``derivations as terms'').

What I am calling a ``proof-theoretic type theory'' is a type theory of the
sort used in the proof assistants Agda, Coq and Idris, whereas the kind of
type theory that I have described in the previous sections, the one based on
meaning explanations, underlies the proof assistant Nuprl.

The proof-theoretic type theories on the one hand are often called
``intensional'' and the meaning-theoretic type theories on the other hand are
usually ``extensional''; these characterizations are certainly true, though I
fear that comparing one of the former with one of the latter is not quite fair,
since there is not any clear analogy to be had. That is to say, the judgement
\ver{M}{P} is a judgement which is added to a logical theory and its meaning is
(briefly) ``$M$ evaluates to a canonical verification of $P$'', whereas $M:P$ cannot
be construed as a judgement added to a logical theory. Instead, it must be
understood as part of a theory which is overlayed atop an existing logical
theory; it is possible to understand the theory which contains the judgement
$M:P$ to be a metatheory, or logical framework, for the theory which contains
the judgement \istrue{P}, which can be construed as the ``object language''.

In short, the judgements \ver{M}{P} and $M:P$ are unrelated to each other in
two respects: firstly, that they have different meanings, and secondly that the
one is at the same level as the judgements of a logical theory, whereas the
latter is a judgement in a theory which is defined over a logical theory.

\section{Martin-L\"of's Logical Framework}

To make this more concrete, let us expound a proof theoretic type theory called
\MLLF, which stands for ``Martin-L\"of's Logical Framework''; in the course of
introducing each type, we will specify which judgement of the underlying
logical theory $\mathcal{L}$ it is meant to interpret. Since we are now
beginning to refer to judgements in multiple theories, where there is
ambiguity, we will use the notation ``$\mathcal{T}$-judgement'' to refer to a
judgement in the theory $\mathcal{T}$, and $\mathcal{T}\vdash\mathcal{J}$ will
be shorthand for ``the $\mathcal{T}$-judgement $\mathcal{J}$ is evident''.

We start with four categorical judgements:\\

\begin{tabular}{c|l}
Judgement & Pronunciation \\ \hline
  \type{\alpha} & $\alpha$ is a type \\
  \type{\alpha=\beta} & $\alpha$ and $\beta$ are equal types \\
  $M:\alpha$ & $M$ is of type $\alpha$ \\
  $M=N:\alpha$ & $M$ and $N$ are equal at type $\alpha$ \\
\end{tabular}\\

But we have not defined the meaning of the judgements; let us do so below:

\begin{quote}
  To know $\type\alpha$ is to know what counts as an object of type $\alpha$,
  and when two such objects are equal.
\end{quote}

For now, we'll leave the question of what is an ``object'' as abstract; in many
cases, types will represent $\mathcal{L}$-judgements and objects will represent
$\mathcal{L}$-derivations.

\begin{quote}
  To know $\type{\alpha=\beta}$ is to know that any object of type $\alpha$ is
  also an object of type $\beta$, and two equal objects of type $\alpha$ are
  equal as objects of type $\beta$ (necessarily presupposing \type\alpha\ and
  \type\beta).
\end{quote}

\begin{quote}
  To know $M:\alpha$ is to know that $M$ is an object of type $\alpha$
  (necessarily presupposing \type\alpha).
\end{quote}

\begin{quote}
  To know $M=N:\alpha$ is to know that $M$ and $N$ are equal objects of type
  $\alpha$ (necessarily presupposing $M:\alpha$\ and $N:\alpha$).
\end{quote}

Next, we will introduce hypothetical judgement in the logical framework; to
avoid confusion, where \hyp{\mathcal{J}}{\mathcal{J}'} is a hypothetical
$\mathcal{L}$-judgement, we will use \lfhyp{\mathcal{J}}{\mathcal{J}'} for
hypothetical judgements in the logical framework. Then,

\begin{quote}
  To know \lfhyp{\mathcal{J}}{\mathcal{J}_1,\dots,\mathcal{J}_n} is to know $\mathcal{J}$ when one knows $\mathcal{J},\dots,\mathcal{J}_n$.
\end{quote}

At this point, we may begin adding types to the logical framework. In practice,
most types which we will introduce in the logical framework will be defined in
terms of a judgement of the logical theory $\mathcal{L}$ which lies below it.
For instance, hypothetical judgement in $\mathcal{L}$ is represented by a
function type in the logical framework, $(x:\alpha)\beta$, whose typehood is
meant to be evident under the following circumstances
\[
  \infer{
    \type{(x:\alpha)\beta}
  }{
    \type\alpha &
    \lfhyp{\type\beta}{x:\alpha}
  }
\]
Or as a hypothetical judgement,
\lfhyp{\type{(x:\alpha)\beta}}{\type\alpha,\lfhyp{\type\beta}{x:\alpha}}.

Now, to know this judgement is to know that under the circumstances we know
what is an object of type $\alpha$ and when two such objects are equal, and
that if we have such an object $x$, we know what an object of type $\beta$ is,
and when two such objects are equal---then we know what an object of type
$(x:\alpha)\beta$ is. To make this evident, then, we will say that under those
circumstances an object of type $(x:\alpha)\beta$ is an object $[x]M$ such that
one knows \lfhyp{M:\beta}{x:\alpha} and \lfhyp{[y/x]M=[z/x]M}{y=z:\alpha};
furthermore, two such objects are equal just when they yield equal outputs for
equal inputs. (The equality is extensional, but since it is with respect to the
definitional equality, this does not break the decidability of the judgement)

Then, for each atomic $\mathcal{L}$-proposition $P$, we can easily define a
type \prf{P}, as follows. Under the circumstances that
$\mathcal{L}\vdash\isprop{P}$, then $\MLLF\vdash\type{\prf{P}}$, since we will
define an object of type \prf{P} to be an $\mathcal{L}$-derivation of
\istrue{P}; beyond reflexivity, further definitional equalities can be added to
reflect the harmony of introduction and elimination rules.

\subsection{What is an ``object''?}
It is time to revisit what it means to be an ``object'' of a type in the
proof-theoretic type theory; we must note how this will necessarily differ from
what it meant to be a ``verification'' of a proposition in the previous
sections. Namely, a verification of a proposition is either a \emph{canonical
verification} of that proposition (and what sort of thing this might be is
known from the presupposition \isprop{P}), or it is a means of getting such
a canonical verification (i.e. a term which evaluates to a canonical
verification).

On the other hand, what we have called an ``object'' of type $P$ is quite
different, since in addition to the possibility that it is a canonical proof of
the judgement \istrue{P}, it may also be \emph{neutral} (i.e. blocked by a
variable); we will call this ``normal'' rather than ``canonical''. Why does
this happen?

In order to keep the judgement $M:A$ analytic (decidable), its meaning
explanation can no longer be based on the idea of the computation of closed
terms to canonical form; instead, we will consider the computation of open
terms (i.e. terms with free variables) to \emph{normal} form. The desire for
$M:A$ to be analytic follows from our intention that it characterize a
\emph{proof theory}: we must be able to recognize a proof when we see one. But
why are closed-term-based meaning explanations incompatible with this goal?
Consider briefly the following judgement:
\[
  \hyp{M(n) \in P}{n\in\naturals}
\]

To know this judgement is to know that $M(n)$ computes to a canonical
verification of $P$ whenever $n$ is a natural number; when $P$'s use of $n$ is
not trivial, this amounts to testing an infinite domain (all of the natural
numbers), probably by means of mathematical induction. The judgement is then
clearly synthetic: to know it is, briefly, to have come up with an (inductive)
argument that $M(N)$ computes to a canonical verification of $P$ at each
natural number $n$.

On the other hand, the judgement \lfhyp{M(n):P}{n:\naturals} must have a
different meaning, one which admits its evidence or refutation purely on
syntactic/analytic grounds. In essence, it is to know that $M(n)$ is a proof of
$P$ for any \emph{arbitrary} object/expression $n$ such that $n:\naturals$
(i.e., the only thing we know about $n$ is that it is of type $\naturals$; we
do not necessarily know that it is a numeral).


\section{A Critique of \textbf{MLLF}}

The type theory which we constructed in the previous section is to be
considered a proof theory for a logic with the judgements \isprop{P},
\istrue{P} and \hyp{\mathcal{J}}{\mathcal{J}_1,\dots,\mathcal{J}_n}. There are
a few reasons to be dissatisfied with this state of affairs, which I shall
enumerate in this section.

\subsection{Lack of Computational Content}

Unlike the type theory in the first chapter, there is no built-in computational
content. In a computational type theory which is defined by the verificationist
meaning explanations, the computational content of terms is understood
immediately by means of the \reduce{M}{M'} relation; that is, computation is
prior to the main judgements because their meaning explanations are defined in
terms of evaluation to canonical form.

On the other hand, in the type theory above we did not give a primitive
reduction relation; instead, we simply permitted the endowement of proofs
with definitional equalities which reflect the harmony of introduction and
elimination rules. That is, if we have known the $\mathcal{L}$-judgement
\istrue{P} by means of an indirect argument (derivation), it must be the case
that this derivation corresponds to a direct one; we reflect this in the
proof theory by defining the indirect derivation to be definitionally equal to
the direct one.

However, this does not amount to computational content being present in terms:
only \emph{post facto} may the definitional equality be construed as giving
rise to computation, through a metamathematical argument which shows that the
definitional equality is confluent and can be used to define a functional
normalization relation.

And this is the reason for the peculiarity of the proof-theoretic meaning
explanations, namely that they do not include phrases like ``evaluates to a
canonical...'', since evaluation may only be understood after taking the
meanings of the judgements (\type\alpha, \type{\alpha=\beta}, $M:\alpha$,
$M=N:\alpha$) as giving rise to a closed formal system which is susceptible to
metamathematical argument: to refer to evaluation in the meaning explanations
for the core judgements, then, would be impredicative.

\subsection{Modularity of definition}

By the same token, the distinction between canonical (direct) and non-canonical
(indirect) proof may not be understood as a core notion in the theory, but must
be understood separately, secondarily. Why is this a problem? It means that the
definition of each type must be made with the full knowledge of the definitions
of every other type; in essence, the open-ended nature of type theory is
obliterated and one is forced into a fixed formal system; this is in addition
to the fact that it causes the epistemic content of \type\alpha for any type
$\alpha$ to be extremely complicated.

To illustrate, let us consider an example which is a type theory that has four
type-formers: trivial truth $\top$, trivial falsity $\bot$, implication
$(\alpha)\beta$, and conjunction \product\alpha\beta; we will then introduce the
following terms to represent proofs: the trivial element $\bullet$,
\emph{reductio ad absurdum} \abort{\alpha}{E}, abstraction $[x]E$, application
$E(E')$, pairing \pair{E}{E'}, and projections \fst{E}, \snd{E}.

If we will try to make the judgement \type\top\ evident, the deficiencies of
formulation will immediately present themselves.

\begin{quote}
To know \type\top\ is to know what counts as an object of type $\top$, and when
two such objects are equal. An object of type $\top$, then, is either the
expression $\bullet$, or an expression \abort{\top}{E} such that we know
$E:\bot$, or an expression $E(E')$ such that we know $E:(\alpha)\top$ and
$E':\alpha$, or an expression \fst{E} such that we know $E:\product\top\beta$
and \type\beta, or an expression \snd{E} such that we know
$E:\product\alpha\top$ and \type\alpha; and we additionally have that $\bullet$
is equal to $\bullet$, and ...
\end{quote}

To save space, we elide the rest of the definition of equality for $\top$; what
we have seen so far already suffices to bring to light two serious problems:

\begin{enumerate}

  \item The meaning explanation of \type\alpha\ is not obvious predicative,
  since it depends upon the meaning of $M:\alpha$.

  \item The definition of any type requires knowledge of the entire syntax of
  the theory. The judgement \type\alpha\ may never be made evident in isolation,
  but must be done with full understanding of all the other types and their definitions.

\end{enumerate}

Furthermore, to extend an existing theory with a new type, the definitions of
every other type must be rewritten to account for the elimination forms of the
new type.

%% As soon as we have begun to consider the computation of open terms to normal
%% form, as opposed to the computation of closed terms to canonical form, a more
%% fine-grained structure of judgements begins to present itself. We'll replace
%% the $M:\alpha$ judgement with two new ones:\\
%%
%% \begin{tabular}{c|l}
%%   \infers{R}{\alpha} & $R$ \textbf{synthesizes} type $\alpha$\\
%%   \checks{M}{\alpha} & $M$ \textbf{checks} type $\alpha$\\
%% \end{tabular}\\
%%
%% A peculiarity of the notation is that \infers{R}{\alpha} is really a synthetic
%% judgement involving the expression $R$ only, where the evidence for the
%% judgement contains $\alpha$. Its meaning explanation is, ``I know of what type $R$
%% is (and by the way, it is $\alpha$)''.  On the other hand, \checks{M}{\alpha} is
%% analytic, and it carries the epistemic content of the old $M:\alpha$ judgement; to
%% know it is to know that $M$ computes to a normal proof of $\alpha$.
%%
%% Let us amend the meaning explanation for the judgements of type theory.
%% \begin{quote}
%%   To know \type{\alpha} is to know what counts as a canonical verification of
%%   $\alpha$, and when two proofs of \type{\alpha} are equal; furthermore, to
%%   know to what is the \emph{purpose} of $\alpha$ (i.e. what is true by virtue
%%   of the truth of $\alpha$).
%% \end{quote}
%%
%% Then the typing judgements are given:
%% \begin{quote}
%%   To know \infers{R}{\alpha} is to know that \reduce{R}{R'} such that $R'$ is a non-canonical proof of a known type, namely $\alpha$.
%% \end{quote}
%% \begin{quote}
%%   To know \checks{M}{\alpha} is to know that \reduce{M}{M'} such that $M'$ is a
%%   normal proof of $\alpha$ (i.e., it is either a canonical proof of $\alpha$ or
%%   it is a neutral term such that one knows \infers{M}{\alpha}).
%% \end{quote}
%%
%% We will restrict the hypothetical judgement to only contain hypotheses of the
%% form \infers{x}{A} where $x$ is a variable, or \type\alpha\ where $\alpha$ is a
%% variable, or further hypothetical judgements:
%% \begin{quote}
%%   To know
%%   \lfhyp{\mathcal{J}}{\mathcal{J}_1,\dots,\mathcal{J}_n} is
%%   to know $\mathcal{J}$ when one knows
%%   $\mathcal{J}_1,\dots,\mathcal{J}_n$.
%% \end{quote}
%%
%% Now, we will show how to make true the judgement \type\alpha\ for a number of
%% types $\alpha$, by defining them. For example, to define conjunction, first we
%% introduce the following syntax: \product\alpha\beta, \pair{M}{N}, \fst{R},
%% \snd{R} subject to the following computation rules:
%% \begin{gather*}
%%   \reduce{\fst{\pair{M}{N}}}{M}
%%   \qquad
%%   \reduce{\snd{\pair{M}{N}}}{N}\\
%%   \infer{
%%     \reduce{\fst{R}}{\fst{R'}}
%%   }{
%%     \reduce{R}{R'} &
%%     \text{$R'$ is neutral}
%%   }
%%   \qquad
%%   \infer{
%%     \reduce{\snd{R}}{\snd{R'}}
%%   }{
%%     \reduce{R}{R'} &
%%     \text{$R'$ is neutral}
%%   }
%% \end{gather*}
%%
%% The, to make the judgement
%% \lfhyp{\type{\product\alpha\beta}}{\type\alpha,\type\beta} evident, let's first
%% expand its meaning explanation.
%% \begin{quote}
%%   To know \lfhyp{\type{\product\alpha\beta}}{\type\alpha,\type\beta} is to know
%%   what counts as a canonical proof of \product\alpha\beta, and when two proofs
%%   of \product\alpha\beta\ are equal, when one knows that what counts as a
%%   canonical proof of $\alpha$ and when two proofs of $\alpha$ are equal, and
%%   one knows what counts as a canonical proof of $\beta$ and when two proofs of
%%   $\beta$ are equal.
%% \end{quote}
%%
%% This is evident, since we will say that a canonical proof of
%% \product\alpha\beta\ is a pair \pair{M}{N} such that we know \checks{M}{\alpha}
%% and \checks{N}{\beta}, and that the equality of pairs is structural. This
%% corresponds to the following inference rules:
%% \[
%%   \infer{
%%     \checks{\pair{M}{N}}{\product\alpha\beta}
%%   }{
%%     \checks{M}{\alpha} &
%%     \checks{N}{\beta}
%%   }
%%   \qquad
%%   \infer{
%%     \pair{M}{N}=\pair{M'}{N'}:\product\alpha\beta
%%   }{
%%     M=M':\alpha &
%%     N=N':\beta
%%   }
%% \]
%%
%% Based on these definitions and computation rules, the following inference rules are justified:
%% \begin{gather*}
%%   \infer{
%%     \infers{\fst{R}}{\alpha}
%%   }{
%%     \infers{R}{\product\alpha\beta}
%%   }
%%   \qquad
%%   \infer{
%%     \infers{\snd{R}}{\beta}
%%   }{
%%     \infers{R}{\product\alpha\beta}
%%   }\\
%%   \infer{
%%     M=\pair{\fst{M}}{\snd{M}}:\product\alpha\beta
%%   }{
%%     \checks{M}{\product\alpha\beta}
%%   }
%% \end{gather*}
\end{document}
