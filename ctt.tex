\documentclass[main.tex]{subfiles}

\begin{document}
\onehalfspacing

\chapter{Computational Type Theories}

As alluded to at the end of the previous chapter, we may add a judgement
\ver{M}{A} which deals directly with the objects $M$ which verify the
propositions (or types) $P$. We will develop a full theory of dependent types in
the sense of Martin-L\"of 1979.

Because we will need to consider the introduction of types which do not have a
trivial (intensional) equality relation, we must first amend the meaning
explanations for some of our judgements, and add a few new forms of judgement.
First, we will refer to \emph{types} rather than \emph{propositions} in order
to emphasize the generality of the theory; in some places, the word \emph{set}
is used instead.

The meaning of hypothetical and general judgement are the same as in the
previous chapter, and so we will not reproduce them here.

The first form of judgement is \framebox{\isset{A}}, and its meaning explanation
is as follows:

\begin{quote}
  To know \isset{A} is to know what counts as a canonical verification of $A$
  and when two such verifications are equal.
\end{quote}

The next form of judgement is \framebox{\ver{M}{A}}, which remains the same as before:

\begin{quote}
  To know \ver{M}{A} (presupposing \isset{A}) is to know an $M'$ such that
  \reduce{M}{M'} and $M'$ is a canonical verification of $A$.
\end{quote}

We'll need to add new judgements for equality (equality of types, and equality
of verifications). For equality of types \framebox{\isset{A=B}}, there are a
number of possible meaning explanations, but we'll use the one that
Martin-L\"of used starting in 1979\footnote{This meaning explanation was
abandoned in Constable's Computational Type Theory for practical reasons.}:

\begin{quote}
  To know \isset{A=B} (presupposing \isset{A} and \isset{B}) is to know
\genj{M}{\hyp{\ver{M}{B}}{\ver{M}{A}}} and
\genj{M}{\hyp{\ver{M}{A}}{\ver{M}{B}}}, and moreover
\genj{M,N}{\hyp{\ver{M=N}{B}}{\ver{M=N}{A}}} and
\genj{M,N}{\hyp{\ver{M=N}{A}}{\ver{M=N}{A}}}.
\end{quote}

In other words, two types are equal when they have the same canonical
verifications, and moreover, the same equality relation over their canonical
verifications.

Now, because we are allowing the definition of types with arbitrary equivalence
relations, we cannot use plain hypothetico-general judgement in the course of
defining our types. For instance, if we were going to try and define the
function type $A\supset B$ in the same way as we did in the previous chapter,
we would permit ``functions'' which are not in fact functional, i.e.\ they do
not take equal inputs to equal outputs. As such, we will need to bake
functionality (also called extensionality) into the definition of functions,
and since we will need this in many other places, we elect to simplify our
definitions by baking it into a single judgement which is meant to be used
instead of hypothetico-general judgement.

First, we will have to give a more restrictive idea of a ``collection of
assumptions''; we will call it a \emph{context}, whose well-formedness is
defined inductively by means of two mutually defined judgements,
\framebox{\isctx{\Gamma}} and \framebox{\fresh{x}{\Gamma}}:

\begin{quote}
  To know \isctx\Gamma\ is to know that \reduce{\Gamma}{\cdot}, or it is to know
  a variable $x$ and an expression $A$ such that \reduce{\Gamma}{\Gamma', \ver{x}{A}}
  and \isset{A}, and \fresh{x}{\Gamma}.
\end{quote}

\begin{quote}
  To know \fresh{x}{\Gamma}\ is to know that \reduce{\Gamma}{\cdot}, or it is to know
  a variable $y$ (which is not $x$) and an expression $A$ such that \reduce{\Gamma}{\Gamma', \ver{x}{A}}
  and \fresh{x}{\Gamma'}.
\end{quote}

In other words contexts are inductively generated by the following grammar:
\begin{gather*}
  \infer{\isctx\cdot}{}
  \qquad
  \infer{
    \isctx{\Gamma,\ver{x}{A}}
  }{
    \isctx\Gamma &
    \isset{A} &
    \fresh{x}{\Gamma}
  }\\
  \infer{\fresh{x}{\cdot}}{}
  \qquad
  \infer{
    \fresh{x}{\Gamma,\ver{y}{A}}
  }{
    \fresh{x}{\Gamma}
  }
\end{gather*}

\end{document}

